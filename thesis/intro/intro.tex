\documentclass{uiucecethesis09} 

\usepackage{standalone} 

\begin{document}
\chapter{Introduction}

This chapter will discuss some of the motivation behind this project, as well as
the goals and distinguishing features of the project. The value of automated
lecture recording as a whole will be described, and an overview will be given of
the techniques to be presented in the remainder of the thesis.

\section{Overview} 

  \subsection{Motivation}

    In the last decade or so, the availability of online educational resources
    has increased significantly. With the launch of MIT OpenCourseWare over 10
    years ago and the now growing popularity of Massive Open Online Courses, the
    internet has become a substantive educational outlet. Additionally, much of
    this media is comprised of lectures given in classrooms or auditoriums to
    real audiences, that have also been recorded and uploaded. Because of this,
    systems for efficiently recording and processing such lectures have become
    more and more important.

    Yet the benefits of such systems can be seen even for students who attend the
    schools holding the lectures. In fact, many times such students are the main
    consideration for these systems, since the availability of a lecture online
    allows students to review that lecture at a later time \cite{bibs}. This can
    free students from the burden of capturing all the material during the actual
    lecture and may allow them to focus more heavily on the concepts being discussed
    rather than writing it down for review later.  

    As it stands, there is a lot that goes into processing many of the open lecture
    resources available online. In addition to ensuring a quality recording of the
    lecture, there is a certain amount of editing and post-processing that usually
    must occur \cite{passivecapture}. Some of this is to improve the flow of the
    material for the end viewer, but some of this may also involve dealing with
    mistakes made by those responsible for filming the lectures. In fact, it has
    been shown that even systems that are not fully automated enjoy better quality
    filming when the filming is attempted automatically and merely monitored by an
    operator (TODO cite).

    In addition to the time saving qualities of automated filming, such techniques
    can also provide substantial cost savings. While the cost of the necessary
    technology for these systems is continually decreasing, the cost of employing a
    person to record and edit the lectures is not. Because of this, prices reported
    for upkeep in such processes can be quite high \cite{Liu:2001}, making any
    opportunity for automation an opportunity for savings as well.  Beyond this,
    because automation can lower costs of producing such media, certain material may
    now be recorded and made available that would not have previously warranted it
    \cite{Bianchi:AVP}. This can help to continue the growth of online media and
    further propel the positive affects of such efforts. 

  \subsection{Focus}
    
    Automatic lecture monitoring as it has been described clearly can involve
    many separate phases. First the lecturer must take any necessary actions to
    initiate the lecture recording. Depending on the system, this may involve
    wearing a certain device to enable tracking, setting up the proper recording
    equipment, or simply pressing some button on a provided interface to
    initiate an automated process. Much of this depends on whether the system is
    designed to be passive or not - that is, whether or not the recorder needs
    to act any differently simply because the lecture is being recorded
    \cite{passivecapture}. 

    In the next step the lecture must actually be recorded -- both in video and
    audio. Again, depending on whether the system is a passive recording system or
    not, this can range from requiring no additional action, to requiring the
    lecturer to take certain actions throughout the presentation to ensure proper
    execution of the monitoring system.

    Finally, there is often a certain amount of post-processing that occurs in order
    to refine the recording.  This can include aligning supplementary materials with
    the video recording, or perhaps slicing up the recording into smaller clips.
    Some systems aim to do this automatically \cite{passivecapture}, but many do
    not.

    Despite the extent of this entire process, in this work we will focus solely on
    the first aspect of this process: the recording of the speaker. More
    specifically, the focus will be on passive recording techniques that utilize
    microphone array processing for locating and tracking the speaker. A camera will
    then be used to film the speaker once the correct location has been determined. 

    In most automated lecture capture systems, computer vision methods are used 
    on the video feeds to track the speaker \cite{Bianchi:AVP},
    \cite{Zhang05anautomated}, \cite{chou:2010}, \cite{Liu:2001}. Microphone arrays
    are then used to locate audience members asking questions, but are not involved
    in the tracking of speakers on stage \cite{Liu:2001}, \cite{Zhang05anautomated}.
    However, microphone array processing techniques could allow for localization of
    the speaker as well, and could help to add robustness to a video tracking system
    by providing additional information.  Moreover, since many systems employing
    computer vision tracking methods have at least two separate cameras -- one for
    following the speaker and one for recording the entire stage to detect movement
    \cite{Bianchi:AVP}, \cite{Liu:2001} -- by using a microphone array to perform
    much, if not all, of the speaker tracking it could be possible to eliminate one
    camera and cut down on the costs of such systems.

\end{document}
