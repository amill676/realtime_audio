\documentclass{uiucecethesis09}
\usepackage{mythesis}

\begin{document}
\chapter{Direction of Arrival Estimation and Stateless Tracking}
  
  This chapter will explore the direction of arrival estimation methods as they 
  were implemented in our system. It will also describe how these estimates were 
  used to perform stateless tracking of speakers. By stateless, we refer to the 
  lack of a state dynamics model for the system or the maintenance of any such 
  external state that would allow for a prediction of the next state. We 
  nevertheless explore methods for smoothing our tracking estimates using 
  alternative approaches.

  \section{Direction of Arrival Estimation}
    We first consider methods for estimating the direction of arrival (DOA) of a 
    sound. As described in \secref{sec:setup}, this equates to estimating the 
    TDOA values given by \eqref{eq:TDOA}.
    \subsection{GCC-PHAT}
      The first method we describe is the Generalized Cross Correlation method 
      \cite{knapp1976generalized}, \cite{brandstein1997robust}, 
      \cite{benesty2010microphone}.
      We seek to find
      \begin{equation} \label{eq:gcc_estimate}
        \hat{\micdelay}_{\micidx, \micjdx} = \argmax_{\micdelay}{\gcc_{\micidx, 
        \micjdx}\of{\micdelay}} \end{equation}
      where
      \begin{align} \label{eq:gcc_dfn}
        \gcc_{\micidx,\micjdx}\of{\micdelay} &= 
        \int\limits_{-\infty}^{\infty}{\gccweightf\Micrec_\micidx^*\offreq\Micrecf{\micjdx} 
        e^{j2\pi\freq\micdelay}} d\freq\\
        %&= \int\limits_{\time^\prime}^{\time^{\prime} + \window} 
        %\micrec_\micidx\of{\time - \micdelay}\micrect{\micjdx} d\time \\
      \end{align}
      is the generalized cross correlation (GCC) function between microphone 
      $\micidx$ and microphone $\micjdx$. Here $\gccweightf$ is a spectrum 
      weighting function that can be used to scale the signal spectra and 
      improve the accuracy of the resulting TDOA estimate. This can be 
      interpreted as an additional set of filters through which the signals pass 
      that can help shape them for better estimation 
      \cite{brandstein1997robust}, \cite{knapp1976generalized}.  We see that for 
      a unity weighting function the GCC is equivalent to the standard cross 
      correlation function

      \begin{align}
        \cc_{\micidx, \micjdx}\of{\micdelay} &= \infint 
        \Micrec_\micidx^*\offreq\Micrecf{\micjdx} e^{j2\pi\freq\micdelay} 
        d\freq\\
        &= \infint \micrec_\micidx\of{\time - \micdelay} \micrect{\micjdx} 
        d\time 
        \label{eq:cc}
      \end{align}

      A variety of weighting functions are available 
      \cite{benesty2010microphone}, \cite{knapp1976generalized}, 
      \cite{brandstein1997robust}, however one of the most popular is the phase 
      transform (PHAT) weighting
      \begin{equation}
        \gccweight_{\text{PHAT}}\of{\freq}= 
        \frac{1}{\abs{\Micrec_\micidx^*\offreq\Micrecf{\micjdx}}}
        \label{eq:phat}
      \end{equation}
      which has been shown to be quite robust in noisy and reverberant 
      environments \cite{omologo93useof}. The combination of the generalized 
      cross correlation method with the phase transform weighting is referred to 
      as GCC-PHAT.

    \subsection{Point Estimates}
      Using the GCC-PHAT method, we can obtain a set of $\tdoaest_{\micidx, 
      \micjdx}$ values by searching for a peak in the resulting GCC function of 
      each necessary microphone pair. This set comprises our TDOA estimates.  
      From this, using \eqref{eq:TDOA} we can set up the following system.
      \begin{equation}
      \frac{1}{\speedsound} \begin{bmatrix} \micpos_1 - \micpos_2 \\ \micpos_1 - 
        \micpos_3 \\ \ \vdots \\ \micpos_1 - \micpos_\nmics \end{bmatrix} 
      \hat{\srcdir} = \begin{bmatrix} \tdoaest_{2} \\
        \tdoaest_{3} \\ \vdots \\ \tdoaest_{\nmics}\end{bmatrix}
      \label{eq:gcc_least_squares}
      \end{equation}
      with
      \begin{equation}
        \tdoaest_{i} = \tdoaest_{1,i}
      \end{equation}
      to be consistent with notation in \eqref{eq:TDOA}. Note that we only 
      consider delays between microphone 1 and all others since other delays 
      become linearly dependent. The solution may be solvable if the number of 
      microphones is one more than the number of dimensions in the search space 
      and the microphones have an appropriate geometry CITE TODO.  Otherwise we 
      must use a least squares solution.

      However, this method has a few negatives. First off, by doing this we will 
      be getting point estimates for the direction of arrival with no measure of 
      likelihood. Therefore, we are making a hard decision about our estimate and 
      if this estimate is used in a larger system, we have made a great 
      commitment at an early stage. This should be avoided if at all possible.  
      Secondly, to actually implement this method we must discretize the 
      $\micdelay$ search space. In doing so, we are inherently discretizing the 
      search space of $\hat{\srcdir}$, which may end up leading to a 
      discretization that is inefficient or ineffective for our purposes.

      A remedy for the first problem could be to aggregate the correlation 
      values for each combination of discrete $\micdelay$ values. Then the 
      magnitude of these correlation values could be used as a likelihood score 
      as we will discuss later.  However, this would be extremely expensive 
      computationally, and in doing so we would be considering many infeasible 
      directions.  Yet, despite this problem, this method is not far off.  In 
      fact if we perform a similar calculation for only combinations of 
      $\micdelay_\micidx$ values that correspond to feasible search directions, 
      we can rectify both of the mentioned problems.

    \subsection{Likelihood Approach}
      As mentioned, the goal is to obtain a likelihood for each feasible 
      direction of arrival. We use methods similar to \cite{dmochowski2008fast}, 
      \cite{DiBiase00ahigh-accuracy}. Unfortunately the space of feasible 
      direction of arrivals is continuous and while attempts to create 
      continuous likelihood models over the search space based on TDOA estimates 
      exist \cite{brandstein1997closed}, we chose to instead discretize the 
      search space. By sampling our search space, we can now determine exactly 
      which TDOAs are of interest.  That is, for each feasible source direction 
      $\dir$ we can calculate $\micdelay_\micidx\of{\dir}$ for any microphone 
      $\micidx$.  We then need only compute the correlation values:
      \begin{align}
      \gcc_{\micidx, \micjdx}\of{\micdelay_\micjdx\of{\dir} - 
      \micdelay_\micidx\of{\dir}} &= \infint \gccweightf 
      \Micrec_\micidx^*\offreq\Micrecf{\micjdx}
      e^{j2\pi\freq\left(\micdelay_\micjdx\of{\dir} - 
      \micdelay_\micidx\of{\dir}\right)} d\freq \\
      \end{align}
      Now to get a likelihood score we average across the different channel 
      pairs, giving
      \begin{equation}
        \lhood_{\text{GCC}}\of{\dir} = \pairsum\shape\of{\gcc_{\micidx, 
        \micjdx}\of{\micdelay_\micjdx\of{\dir} - \micdelay_\micidx\of{\dir}}}
      \end{equation}
      where $\shape\of{x} = x^k$ is a likelihood shaping function, used to 
      sharpen peaks and reduce sidelobes \cite{ward2002particle}.  
      
      \begin{figure}[h]
        \centering
        \includegraphics[width=\textwidth]{figures/gcc_likelihoods.png}
        \caption{Normalized GCC Values for various values of $k$ in the shaping 
          function $\shape\of{x} = x^k$. We see the speak sharpen as the value 
        of $k$ increases from 1 to 5.}
        \label{fig:gcc_shapes}
      \end{figure}

    \subsection{Beamforming and SRP-PHAT}
      The method described is also sometimes referred to as SRP-PHAT (where SRP 
      stands for Steered Response Power)  when the PHAT weighting function is 
      used to calculate the GCC \cite{dmochowski2008fast}.  This results from 
      the fact that we are calculating the energy in the scaled Crosspower 
      spectrum \cite{omologo93useof} after steering the array in each feasible 
      direction $\dir$. This suggests that this method can be viewed in a 
      beamforming context.

      Consider using a delay and sum beamformer to steer the signal in each 
      feasible direction $\dir$ and then taking the power of the corresponding 
      signal. 

      \begin{align}
        \lhood_{\text{DS}}\of{\dir} &= \infint \left[\micsum 
        \micrec_\micidx\of{\time + \micdelay_\micidx\of{\dir}} \right]^2 d\time 
        \\
        &= \micsum \micsumk{\micjdx} \infint \micrec_\micidx\of{\time + 
          \micdelay_\micidx\of{\dir}} \micrec_\micjdx\of{\time + 
            \micdelay_\micjdx\of{\dir}} d\time \\
        &= \micsum \micsumk{\micjdx} \infint \micrec_\micidx\of{\time + 
          \micdelay_\micidx\of{\dir}-\micdelay_\micjdx\of{\dir} } 
          \micrec_\micjdx\oftime d\time
        \label{eq:srp_phat} \\
        &= \micsum \micsumk{\micjdx} \infint \Micrec_\micidx^*\offreq 
        \Micrecf{\micjdx} e^{j2\pi\freq\left(\micdelay_\micjdx\of{\dir} - 
        \micdelay_\micidx\of{\dir}\right)} d\freq \\
        &= 2\lhood_{\text{GCC}}\of{\dir} + \micsum \infint 
        \Micrec_\micidx^*\offreq\Micrecf{\micidx} d\freq
      \end{align}

      We see that if we use $\shape\of{x} = x$ and add in a weighting function 
      $\gccweightf$ we get that the result is equal to our earlier likelihood 
      plus the total signal energy. It can be shown that these two likelihood 
      scores are closely related to a Bayesian likelihood of the signal given 
      the prospective direction \cite{Birchfield02fastbayesian}.

      Just as a spectrum weighting function $\gccweightf$ was added with the GCC 
      method, a weighting function for combining frequencies can be added here . 
      While there again are various weighting functions, the phase transform is 
      often used, leading to the denomination of this method as SRP-PHAT 
      \cite{DiBiase00ahigh}. There also exist analogous SRP algorithms for 
      beamforming methods other than delay and sum.  Good coverage of these 
      methods is given in \cite{tashev2009sound}
      TODO FIGURE

  \section{Tracking}
    \subsection{Raw DOA Estimates}
      The most obvious
        
      






\end{document}
